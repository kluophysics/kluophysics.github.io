\documentclass[11pt]{article}
% \documentclass{assignment}

\usepackage[margin=1in]{geometry} 
\usepackage{amsmath,amsthm,amssymb, graphicx, multicol, array}
\usepackage{ctex} 
\usepackage{fancyhdr}
\usepackage{fontspec}
\usepackage{fourier-orns} % 

% % http://kuing.infinityfreeapp.com/forum.php?mod=viewthread&tid=527&i=1
% \let\oldjuhao=。
% \catcode`\。=\active
% \newcommand{。}{\ifmmode\text{\oldjuhao}\else\oldjuhao\fi}

% \let\olddouhao=,
% \catcode`\,=\active
% \newcommand{,}{\ifmmode\text{\olddouhao}\else\olddouhao\fi}



\newcommand{\N}{\mathbb{N}}
\newcommand{\Z}{\mathbb{Z}}
 
\theoremstyle{remark}
\newtheorem{problem}{}


% \newenvironment{problem}[2][]{\begin{trivlist}
% \item[\hskip \labelsep {\bfseries #1}\hskip \labelsep {\bfseries #2}]}{\end{trivlist}}

\renewcommand{\headrule}{%
\vspace{-8pt}\hrulefill
\raisebox{-2.0pt}{\quad\decofourleft\decotwo\decofourright\quad}\hrulefill}


\newcommand{\duedate}{\zhdate{2024/9/23}}
\pagestyle{fancy}

% \fancyhf{}
% \fancyhead[LE]{\nouppercase{\rightmark\hfill\leftmark}}
% \fancyhead[RO]{\nouppercase{\leftmark\hfill\rightmark}}

%... then configure it.
\fancyhead{} % clear all header fields
% \fancyhead[LE]{\textbf{PHYS}}
% \fancyhead[L]{\textbf{\leftmark}}
% \fancyhead[L]{\chaptermark{}}
% \fancyhead[L]{\sectionmark{}}
% \fancyhead[LR]{\textbf{\leftmark}}
\fancyhead[L]{数学物理方法}
\zhnumsetup{style={Traditional,Financial}}
\fancyhead[R]{课后习题\zhnumber{1}}

% \fancyhead[RO]{\textbf{\rightmark}}

% \fancyhead[LO]{\textbf{Kai Luo}}
% \fancyhead[LO]{罗凯}

% \fancyhead[LE]{\nouppercase{\rightmark\hfill\leftmark}}
% \fancyhead[RO]{\nouppercase{\leftmark\hfill\rightmark}}

\fancyfoot{} % clear all footer fields
% \fancyfoot[CE,CO]{\rightmark}
% \fancyfoot[LO,CE]{}
% \fancyfoot[CO,RE]{Nanjing University of Science and Technology}
\fancyfoot[C]{\thepage}
\fancyfoot[L]{截止日期\duedate} % clear all footer fields
\fancyfoot[R]{任课教师:罗凯} % clear all footer fields

% \fancyfoot[CE,O]{\thechapter}



\begin{document}
\renewcommand{\labelenumi}{(\arabic{enumi})}
\renewcommand{\labelenumii}{(\arabic{enumi}.\arabic{enumii})}



% \title{习题01}
% \author{截止日期: }
% \author{罗凯\\
% 数学物理方法}
% \date{截止日期:}
% \maketitle
 
% \begin{problem}{1.1}
% 找出满足方程\[z - 2 = 3 \frac{1+ \imath t}{1-\imath t}, -\infty < t <\infty\]的所有点$z$.
% \end{problem}

% \begin{problem}{1.1}
% 写出下列复数的实部,虚部,模和辐角.
% \begin{enumerate}
%   \item $1 + \imath \sqrt{3}$,
%   \item $\frac{1-\imath}{1+\imath}$,
%   \item $e^{z}$,
%   \item $e^{1+\imath}$,
%   \item $e^{\imath \varphi(x)}$, $\varphi(x)$是实数$x$的实函数.
% \end{enumerate}
% \end{problem}

\begin{problem}
  写出下列复数的实部,虚部,模和辐角.
  \begin{enumerate}
    \item $1 + \imath \sqrt{3}$,
    \item $\frac{1-\imath}{1+\imath}$,
    \item $e^{z}$,
    \item $e^{1+\imath}$,
    \item $e^{\imath \varphi(x)}$, $\varphi(x)$是实数$x$的实函数.
  \end{enumerate}
  \end{problem}

\begin{problem}
  证明以下规律:
  \begin{enumerate}
    \item 复数的加法和乘法满足结合律,即 
    \begin{itemize}
      \item $(z_1 + z_2 ) + z_3 = z_1 +( z_2 + z_3)$
      \item $(z_1 \cdot z_2) \cdot z_3 = z_1 \cdot (z_2 \cdot z_3)$
    \end{itemize}

    \item 复数乘积的共轭等于复数共轭的乘积,即$(z_1 \cdot z_2)^* = z_1^* z_2^*$.
  \end{enumerate}  
\end{problem}

\begin{problem}
  找出复平面上满足方程\[|z - \imath a|  = \lambda | z + \imath a|, \lambda > 0,\]
  的所有点$z = x + \imath y$.请分$\lambda$的三种情况(1) $\lambda<1$, (2) $\lambda >1$,(3) $\lambda = 1$分别绘制图形.
\end{problem}

\begin{problem}
  若$|z|=1$, $a,b$为任意复数,试证明
  $$
  \left| \frac{a z + b}{\bar{b} z + \bar{a}} \right| = 1 .
  $$
\end{problem}

\begin{problem}
  试用$\cos\varphi, \sin\varphi$ 表示$\cos 4\varphi$.
\end{problem}

\begin{problem}
  求下列方程的根,并在复平面上画出它们的位置.
  \begin{enumerate}
    \item $z^4 + 1 = 0$,
    \item $z^2 + 2 z \cos \lambda + 1 = 0, \quad 0 < \lambda < \pi$.
  \end{enumerate}
\end{problem}

\begin{problem}
验证
$$
\tan^{-1}(z)=\frac{\imath }{2}[\ln (1-\imath z)-\ln (1+\imath z)] .
$$
\end{problem}


\begin{problem}
  试证明极坐标下的柯西-黎曼方程
  $$
    \left\{\begin{array}{l}
    \frac{\partial u}{\partial \rho}=\frac{1}{\rho} \frac{\partial v}{\partial \varphi} ,\\
    \frac{1}{\rho} \frac{\partial u}{\partial \varphi}=-\frac{\partial v}{\partial \rho} .
    \end{array}\right.
  $$
\end{problem}

% \begin{problem}
%   已知解析函数 $f(z)$  的实部 $u(x, y)$或虚部$ v(x, y)$, 求该解析函数.
%   \begin{enumerate}
%     \item $u = e^x \sin{y}$,
%     \item $u = x^2 - y^2 + xy$, $f(0) = 0$.
%   \end{enumerate}
% \end{problem}

% \begin{proof}[Solution]
% Write a solution here
% \begin{align*}
% x &= y
% \end{align*}
% \begin{tabular}{| >{\centering\arraybackslash}m{1in} | >{\centering\arraybackslash}m{1in} | >{\centering\arraybackslash}m{1in} | >{\centering\arraybackslash}m{1in} |>{\centering\arraybackslash}m{1in} |}
% \hline 
%   \textbf{A} & \textbf{B} & \textbf{C} & \textbf{D} &\textbf{E} \\[8pt]
%   \hline
%   a & b & c & d & e \\[8pt]
%   \hline
%   f & g &h & i & j \\[8pt]
%   \hline
% \end{tabular}
% \end{proof}

\end{document}